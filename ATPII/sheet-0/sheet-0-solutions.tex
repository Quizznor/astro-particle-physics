\documentclass{scrartcl}
\usepackage{amsmath}
\usepackage[ngerman]{babel}
\usepackage{listingsutf8}
\usepackage{xcolor}
\usepackage[utf8]{inputenc}

\definecolor{mywhite}{HTML}{ebebeb}
\definecolor{mygrey}{HTML}{757575}
\definecolor{mygreen}{HTML}{00ab28}
\definecolor{mypurple}{HTML}{6b0ead}

\lstset{%
numbers=left,                      % where to draw (if at all) line numbers
showspaces=false,                  % whitespaces in the code appear as _
backgroundcolor=\color{mywhite},   % choose the background color
basicstyle=\footnotesize,          % size of fonts used for the code
breaklines=true,                   % automatic line breaking only at whitespace
captionpos=b,                      % sets the caption-position to bottom
commentstyle=\color{mygrey},       % comment style
escapeinside={\%*}{*)},            % if you want to add LaTeX within your code
keywordstyle=\color{purple},       % keyword style
stringstyle=\color{mygreen},       % string literal style
literate={°}{{$^\circ$}}1          % custom characters for file encoding
}

\setlength{\parindent}{0ex}

\title{Sheet 0 Solutions}
\author{Paul Filip}

\begin{document}

\maketitle

\section*{Exercise 1}

\subsection*{a)}
Das Virgo-Cluster befindet sich bei einer Rektaszension von 12h 27m und Deklination von 12$^\circ$ 43'.
Geben Sie Koordinaten in Radiant und Grad im aquatorialen und im galaktischen Koordinatensystem an.

\vspace{0.5cm}
\lstinputlisting[language=python]{sheet-0a.py}

\subsection*{b)}
Der Frühlingspunkt bestimmt die Koordinatenmitte im eklitpischen und äquatorialen Koordinatensystem.
Erklären Sie die Definition des Frühlingspunktes (ggf. mit Skizze) und geben Sie dessen Koordinaten
im galaktischen und supergalaktischen Koordinatensystem an.

\vspace{0.5cm}
\textbf{Der Frühlingspunkt ist einer der beiden Schnittpunkte des Himmelsequators mit der Ekliptik.
Genauer handelt es sich dabei um den Schnittpunkt, in dem die Sonne järhlich am 21. März steht.}

\vspace{0.5cm}
\lstinputlisting[language=python]{sheet-0b.py}

\subsection*{c)}
Erstellen Sie je ein Schaubild in äquatorialen, galaktischen und supergalaktischen Koordinaten, auf dem
der Frühlingspunkt, das Virgo-Cluster, das galaktische Zentrum und die galaktische Ebene erkennbar sind.

\end{document}
